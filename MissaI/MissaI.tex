% !TEX TS-program = lualatex
% !TEX encoding = UTF-8

\documentclass[12pt]{article} % use larger type; default would be 10pt

\usepackage{fontspec}
\usepackage{graphicx} % support the \includegraphics command and options
\usepackage{geometry} % See geometry.pdf to learn the layout options. There are lots.
\geometry{a4paper} % or letterpaper (US) or a5paper or....
\usepackage{gregoriotex} % for gregorio score inclusion
\usepackage{fullpage} % to reduce the margins


\begin{document}

% The title:
\begin{center}\begin{huge}\textsc{Missa I.}\end{huge}\end{center}
\begin{large}\textsc{Lux et origo}\hfill{\em V~době velikonoční}\end{large}

% Here we set the space around the initial.
\setspaceafterinitial{2.2mm plus 0em minus 0em}
\setspacebeforeinitial{2.2mm plus 0em minus 0em}

% Here we set the initial font.
\def\greinitialformat#1{{\fontsize{43}{43}\selectfont #1}}

% We set red lines here, comment it if you want black ones.
\redlines

% We type a text in the top right corner of the score:
%\commentary{{\small \emph{Cf. Is. 30, 19 . 30 ; Ps. 79}}}

% and finally we include the score. The file must be in the same directory as this one.
\vspace{0.5 cm}
\includescore{Kyrie}

\vspace{1 cm}
\includescore{Gloria}

\vspace{1 cm}
\includescore{Sanctus}

\vspace{1 cm}
\includescore{AgnusDei}

\vspace{0.5 cm}
\commentary{{\small \emph{Od Bílé soboty do soboty oktávu velikonočního včetně}}}

\vspace{0.5 cm}
\includescore{Ite-a}

\vspace{0.5 cm}

\commentary{{\small \emph{Od neděle 1. po Velikonocích do Božího hodu Svatodušního včetně}}}
\vspace{0.5 cm}
\includescore{Ite-b}

\end{document}
