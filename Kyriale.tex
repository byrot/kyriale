% !TEX TS-program = lualatex
% !TEX encoding = UTF-8

\documentclass[12pt]{article} % use larger type; default would be 10pt

\usepackage{fontspec}
\usepackage{graphicx} % support the \includegraphics command and options
\usepackage{geometry} % See geometry.pdf to learn the layout options. There are lots.
\geometry{a4paper} % or letterpaper (US) or a5paper or....
\usepackage{gregoriotex} % for gregorio score inclusion
\usepackage{fullpage} % to reduce the margins

% custom settings for section headlines
\usepackage{titlesec}
\titleformat{\section}[block]{\huge\sc\filcenter}{}{1em}{}

% macro info for informational subtitles
\newcommand{\info}[2]{\large\textsc{#1}\hfill{\em #2}}

% custom settings for ToC: no numbers, title, font style, dots
\usepackage{tocloft}
\setcounter{secnumdepth}{0}
\renewcommand{\contentsname}{\hfil\huge\sc Obsah\hfil\par\vspace{1em}}
\renewcommand{\cftsecleader}{\cftdotfill{\cftdotsep}}
\renewcommand{\cftsecfont}{\rm}
\renewcommand{\cftsecpagefont}{\rm}

% define symbols for Versicle and Response
\usepackage{fontspec}
\newcommand{\textjuni}[1]{{\fontspec{Junicode}#1}}
\newcommand{\V}{\textjuni{\char"2123.}}
\newcommand{\R}{\textjuni{\char"211F.}}



\begin{document}


%%%%%%%%%%%%%%%%%%%%%%%%%%%%% TITLE PAGE %%%%%%%%%%%%%%%%%%%%%%%%%%%%%%%%%
\begin{titlepage}
  \begin{center}

    { \vspace*{60 mm} }
    \begin{Huge}\textbf{K Y R I A L E}\end{Huge}

    \vspace{30 mm}
    \begin{Large}{Latinské mešní zpěvy v~gregoriánské notaci}\end{Large}

    \vspace{110 mm}
    \begin{large}{A.~D.~2015}\end{large}

  \end{center}
\end{titlepage}



%%%%%%%%%%%%%%%%%%%%%%%%%%%%% GLOBAL SETTINGS %%%%%%%%%%%%%%%%%%%%%%%%%%%%

% Here we set the space around the initial.
\setspaceafterinitial{2.2mm plus 0em minus 0em}
\setspacebeforeinitial{2.2mm plus 0em minus 0em}

% Here we set the initial font.
\def\greinitialformat#1{{\fontsize{43}{43}\selectfont #1}}

% We set red lines here, comment it if you want black ones.
\redlines





%%%%%%%%%%%%%%%%%%%%%%%%%%%%% MISSA I. %%%%%%%%%%%%%%%%%%%%%%%%%%%%%%%%%%%

\section{Ordinárium I.}
\info{Lux et origo}{V~době velikonoční}

\vspace{0.5 cm}
\includescore{MissaI/Kyrie}

\vspace{1 cm}
\includescore{MissaI/Gloria}

\vspace{1 cm}
\includescore{MissaI/Sanctus}

\vspace{1 cm}
\includescore{MissaI/Agnus}

\vspace{0.5 cm}
\commentary{{\small \em{Od Bílé soboty do soboty oktávu velikonočního včetně}}}

\vspace{0.5 cm}
\includescore{MissaI/Ite-a}

\vspace{0.5 cm}

\commentary{{\small \emph{Od neděle 1. po Velikonocích do Božího hodu Svatodušního včetně}}}
\vspace{0.5 cm}
\includescore{MissaI/Ite-b}



\vspace{1 cm}

%%%%%%%%%%%%%%%%%%%%%%%%%%%%% MISSA IV. %%%%%%%%%%%%%%%%%%%%%%%%%%%%%%%%%%

\section{Ordinárium IV.}
%\begin{large}\textsc{Lux et origo}\hfill{\em V~době velikonoční}\end{large}

\vspace{0.5 cm}
\includescore{MissaIV/Kyrie}

\vspace{1 cm}
\includescore{MissaIV/Gloria}

\vspace{1 cm}
\includescore{MissaIV/Sanctus}

\vspace{1 cm}
\includescore{MissaIV/Agnus}

\vspace{1 cm}
\includescore{MissaIV/Ite}




\vspace{1 cm}

%%%%%%%%%%%%%%%%%%%%%%%%%%%%% MISSA IX. %%%%%%%%%%%%%%%%%%%%%%%%%%%%%%%%%%

\section{Ordinárium IX.}
%\begin{large}\textsc{Lux et origo}\hfill{\em V~době velikonoční}\end{large}

\vspace{0.5 cm}
\includescore{MissaIX/Kyrie}

\vspace{1 cm}
\includescore{MissaIX/Gloria}

\vspace{1 cm}
\includescore{MissaIX/Sanctus}

\vspace{1 cm}
\includescore{MissaIX/Agnus}

\vspace{1 cm}
\includescore{MissaIX/Ite}


\newpage
\tableofcontents

\end{document}
